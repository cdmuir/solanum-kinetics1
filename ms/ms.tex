% Options for packages loaded elsewhere
\PassOptionsToPackage{unicode}{hyperref}
\PassOptionsToPackage{hyphens}{url}
\PassOptionsToPackage{dvipsnames,svgnames,x11names}{xcolor}
%
\documentclass[
  letterpaper,
  DIV=11,
  numbers=noendperiod]{scrartcl}

\usepackage{amsmath,amssymb}
\usepackage{iftex}
\ifPDFTeX
  \usepackage[T1]{fontenc}
  \usepackage[utf8]{inputenc}
  \usepackage{textcomp} % provide euro and other symbols
\else % if luatex or xetex
  \usepackage{unicode-math}
  \defaultfontfeatures{Scale=MatchLowercase}
  \defaultfontfeatures[\rmfamily]{Ligatures=TeX,Scale=1}
\fi
\usepackage{lmodern}
\ifPDFTeX\else  
    % xetex/luatex font selection
  \setmainfont[]{Times New Roman}
\fi
% Use upquote if available, for straight quotes in verbatim environments
\IfFileExists{upquote.sty}{\usepackage{upquote}}{}
\IfFileExists{microtype.sty}{% use microtype if available
  \usepackage[]{microtype}
  \UseMicrotypeSet[protrusion]{basicmath} % disable protrusion for tt fonts
}{}
\makeatletter
\@ifundefined{KOMAClassName}{% if non-KOMA class
  \IfFileExists{parskip.sty}{%
    \usepackage{parskip}
  }{% else
    \setlength{\parindent}{0pt}
    \setlength{\parskip}{6pt plus 2pt minus 1pt}}
}{% if KOMA class
  \KOMAoptions{parskip=half}}
\makeatother
\usepackage{xcolor}
\setlength{\emergencystretch}{3em} % prevent overfull lines
\setcounter{secnumdepth}{-\maxdimen} % remove section numbering
% Make \paragraph and \subparagraph free-standing
\ifx\paragraph\undefined\else
  \let\oldparagraph\paragraph
  \renewcommand{\paragraph}[1]{\oldparagraph{#1}\mbox{}}
\fi
\ifx\subparagraph\undefined\else
  \let\oldsubparagraph\subparagraph
  \renewcommand{\subparagraph}[1]{\oldsubparagraph{#1}\mbox{}}
\fi


\providecommand{\tightlist}{%
  \setlength{\itemsep}{0pt}\setlength{\parskip}{0pt}}\usepackage{longtable,booktabs,array}
\usepackage{calc} % for calculating minipage widths
% Correct order of tables after \paragraph or \subparagraph
\usepackage{etoolbox}
\makeatletter
\patchcmd\longtable{\par}{\if@noskipsec\mbox{}\fi\par}{}{}
\makeatother
% Allow footnotes in longtable head/foot
\IfFileExists{footnotehyper.sty}{\usepackage{footnotehyper}}{\usepackage{footnote}}
\makesavenoteenv{longtable}
\usepackage{graphicx}
\makeatletter
\def\maxwidth{\ifdim\Gin@nat@width>\linewidth\linewidth\else\Gin@nat@width\fi}
\def\maxheight{\ifdim\Gin@nat@height>\textheight\textheight\else\Gin@nat@height\fi}
\makeatother
% Scale images if necessary, so that they will not overflow the page
% margins by default, and it is still possible to overwrite the defaults
% using explicit options in \includegraphics[width, height, ...]{}
\setkeys{Gin}{width=\maxwidth,height=\maxheight,keepaspectratio}
% Set default figure placement to htbp
\makeatletter
\def\fps@figure{htbp}
\makeatother
% definitions for citeproc citations
\NewDocumentCommand\citeproctext{}{}
\NewDocumentCommand\citeproc{mm}{%
  \begingroup\def\citeproctext{#2}\cite{#1}\endgroup}
\makeatletter
 % allow citations to break across lines
 \let\@cite@ofmt\@firstofone
 % avoid brackets around text for \cite:
 \def\@biblabel#1{}
 \def\@cite#1#2{{#1\if@tempswa , #2\fi}}
\makeatother
\newlength{\cslhangindent}
\setlength{\cslhangindent}{1.5em}
\newlength{\csllabelwidth}
\setlength{\csllabelwidth}{3em}
\newenvironment{CSLReferences}[2] % #1 hanging-indent, #2 entry-spacing
 {\begin{list}{}{%
  \setlength{\itemindent}{0pt}
  \setlength{\leftmargin}{0pt}
  \setlength{\parsep}{0pt}
  % turn on hanging indent if param 1 is 1
  \ifodd #1
   \setlength{\leftmargin}{\cslhangindent}
   \setlength{\itemindent}{-1\cslhangindent}
  \fi
  % set entry spacing
  \setlength{\itemsep}{#2\baselineskip}}}
 {\end{list}}
\usepackage{calc}
\newcommand{\CSLBlock}[1]{\hfill\break\parbox[t]{\linewidth}{\strut\ignorespaces#1\strut}}
\newcommand{\CSLLeftMargin}[1]{\parbox[t]{\csllabelwidth}{\strut#1\strut}}
\newcommand{\CSLRightInline}[1]{\parbox[t]{\linewidth - \csllabelwidth}{\strut#1\strut}}
\newcommand{\CSLIndent}[1]{\hspace{\cslhangindent}#1}

\usepackage{booktabs}
\usepackage{longtable}
\usepackage{array}
\usepackage{multirow}
\usepackage{wrapfig}
\usepackage{float}
\usepackage{colortbl}
\usepackage{pdflscape}
\usepackage{tabu}
\usepackage{threeparttable}
\usepackage{threeparttablex}
\usepackage[normalem]{ulem}
\usepackage{makecell}
\usepackage{xcolor}
\KOMAoption{captions}{tableheading}
\usepackage{hyperref}
\addtokomafont{disposition}{\rmfamily}
\usepackage{siunitx}
\newcommand{\Aamphi}{$A_{\mathrm{amphi}}$}
\newcommand{\Ahypo}{$A_{\mathrm{hypo}}$}
\newcommand{\agcurve}{$A \textendash g_\text{sw}$}
\newcommand{\ca}{$C_\mathrm{a}$}
\newcommand{\caequals}[1]{$C_\mathrm{a} = \SI{#1}{\micro\mol\raiseto{-1}\mol}$}
\newcommand{\cabetween}[2]{#1 < $C_\mathrm{a} < \SI{#2}{\micro\mol\raiseto{-1}\mol}$}
\newcommand{\fgmax}{$f_\text{gmax}$}
\newcommand{\gcl}{$l_\text{gc}$}
\newcommand{\gmax}{$g_\text{max}$}
\newcommand{\gmaxratio}{$g_\text{max,ratio}$}
\newcommand{\gmaxshade}{$g_\text{max,shade}$}
\newcommand{\gmaxsun}{$g_\text{max,sun}$}
\newcommand{\gsw}{$g_\text{sw}$}
\newcommand{\loggsw}{$\log(g_\text{sw})$}
\newcommand{\logA}{$\log(A)$}
\newcommand{\ppfd}{$\mathrm{PPFD}$}
\newcommand{\ppfdequals}[1]{$\mathrm{PPFD} = \SI{#1}{\micro\mole\per\meter\squared\per\second}$}
\newcommand{\ppfdqty}[1]{$\SI{#1}{\micro\mole\per\meter\squared\per\second}$}
\newcommand{\rh}{$\mathrm{RH}$}
\newcommand{\rhequals}[1]{$\mathrm{RH} = #1\%$}
\newcommand{\tleafequals}[1]{$T_\mathrm{leaf} = \SI{#1}{\degreeCelsius}$}
\usepackage{setspace}
\usepackage[modulo]{lineno}
\usepackage{physics}
\usepackage{pifont}
\newcommand{\cmark}{\ding{51}}
\makeatletter
\@ifpackageloaded{caption}{}{\usepackage{caption}}
\AtBeginDocument{%
\ifdefined\contentsname
  \renewcommand*\contentsname{Table of contents}
\else
  \newcommand\contentsname{Table of contents}
\fi
\ifdefined\listfigurename
  \renewcommand*\listfigurename{List of Figures}
\else
  \newcommand\listfigurename{List of Figures}
\fi
\ifdefined\listtablename
  \renewcommand*\listtablename{List of Tables}
\else
  \newcommand\listtablename{List of Tables}
\fi
\ifdefined\figurename
  \renewcommand*\figurename{Figure}
\else
  \newcommand\figurename{Figure}
\fi
\ifdefined\tablename
  \renewcommand*\tablename{Table}
\else
  \newcommand\tablename{Table}
\fi
}
\@ifpackageloaded{float}{}{\usepackage{float}}
\floatstyle{ruled}
\@ifundefined{c@chapter}{\newfloat{codelisting}{h}{lop}}{\newfloat{codelisting}{h}{lop}[chapter]}
\floatname{codelisting}{Listing}
\newcommand*\listoflistings{\listof{codelisting}{List of Listings}}
\makeatother
\makeatletter
\makeatother
\makeatletter
\@ifpackageloaded{caption}{}{\usepackage{caption}}
\@ifpackageloaded{subcaption}{}{\usepackage{subcaption}}
\makeatother
\ifLuaTeX
  \usepackage{selnolig}  % disable illegal ligatures
\fi
\usepackage{bookmark}

\IfFileExists{xurl.sty}{\usepackage{xurl}}{} % add URL line breaks if available
\urlstyle{same} % disable monospaced font for URLs
\hypersetup{
  pdftitle={Guard cell size and pore aperture influence stomatal closure kinetics},
  pdfauthor={Christopher D. Muir\^{}\{1,2*\}; Wei Shen Lim\^{}\{1\}},
  colorlinks=true,
  linkcolor={blue},
  filecolor={Maroon},
  citecolor={Blue},
  urlcolor={Blue},
  pdfcreator={LaTeX via pandoc}}

\title{Guard cell size and pore aperture influence stomatal closure
kinetics}
\author{Christopher D. Muir\(^{1,2*}\) \and Wei Shen Lim\(^{1}\)}
\date{}

\begin{document}
\maketitle

\begin{center}
$^1$School of Life Sciences, University of Hawaiʻi; Mānoa, Hawaiʻi, USA\\
$^2$Department of Botany, University of Wisconsin, Madison; Wisconsin, USA\\
\medskip
*Corresponding author: cdmuir@wisc.edu\\
\medskip
ORCID: Christopher D. Muir — \href{https://orcid.org/0000-0003-2555-3878}{0000-0003-2555-3878}\\
\medskip
Classification: Biological Sciences, Ecology\\
\medskip
Keywords: amphistomy, leaf gas exchange, \textit{Solanum}, stomatal kinetics, stomatal size
\end{center}

Reminder: Halliwell et al.~2025 paper has useful info on multiresponse
models. I might use Van de Pol and Wright (2009) method for within
versus between slope estimation. This is in the phylogenetic comparative
example in the brms vignette too. Actually, this is hard to implement
because I would have to calculate individual residuals for each light
treatment combination and then means of those, etc.

Westoby et al (2023) (10.1111/1365-2745.14150) is similar to Halliwell.
Maybe another way to think about it using the mutliresponse approach.

\subsection{Abstract}\label{abstract}

Guard cell size and stomatal density vary over X and X orders of
magnitude among extant vascular land plants, yet the adaptive
significance of much of this variation remains unclear. The evolution of
guard cell size in particular is poorly understood and may even be
constrained by nonadaptive features like genome size. One hypothesis is
that natural selection may favor smaller guard cells to increase the
rate at which stomatal conductance responds to fluctuating environmental
conditions. A related hypothesis is that operational stomatal
conductance (gop) is often \(\approx 25\%\) of its theoretical maximum
(gmax) because at this stomatal aperture, guard cell volume is poised to
change rapidly with changes in turgor pressure. The support for both
hypothesis is limited and mixed, in part because they have not been
tested together, even though both guard cell size and the ratio of gop
to gmax (gop:gmax) should influence stomatal kinetics. We measured
stomatal closure kinetics in response to an abrupt increase in vapor
pressure deficit (VPD) among 29 diverse wild tomato populations in the
genus \textit{Solanum}. Both smaller guard cell size and lower gop:gmax
were associated with faster stomatal closure kinetics, but at different
levels of biological organization. Guard cell size explained X\% of the
variance in stomatal kinetics among populations, whereas gop:gmax
explained X\% of the variance among individuals within populations.
{[}neither result would have been supported if analyzed in isolation -
is this true?{]}. We conclude that guard cell size may respond to
selection on stomatal kinetics over evolutionary time, but actual
kinetics are modulated by a tradeoff between maximizing gop and
minimizing the time required for stomatal closure. Putting these
hypotheses together with stabilizing selection on gop may explain why
stomatal size and density negatively covary among species.

\subsection{Introduction}\label{introduction}

NOTE: I need to develop a consistent terminology - stomatal response
rate? speed? kinetics?

In nature, change is the only constant. One way that vascular plants
(tracheophytes) cope with change is by adjusting stomatal pore aperture
to optimize the trade-off between carbon gain and water loss. In the
absence of physical limits and energetic tradeoffs, natural selection
would favor plants that could instantaneously adjust stomatal
conductance (\gsw) to perfectly track dynamic environmental conditions.
In reality, stomatal responses take time, creating a lag between actual
and optimal stomatal aperture. X ions must be pumped across \ldots{} and
active responses depend on multiple feedback loops involving gene
expression/ABA synthesis?/etc (Buckley). {[}illustrative empirical
example or real numbers illustrating the problem{]}. Changes in stomatal
aperture could be faster if there were more channels (?) per area, but
this would require more energy (ATP?) and space allocated to XX proteins
in the guard cell membrane.

It might seem that natural selection should favor faster stomatal
responses to get closer to the ideal of instantaneous optimization.
Under this assumption, variation in stomatal kinetics would depend on
how strong directional selection for faster responses is counterbalanced
by costs of faster stomata. We discuss multiple, nonmutually exclusive
costs below {[}WHERE?{]}. However, once models account for inevitable
time lags, natural selection can favor slower responses when this
prevents overreacting to unpredictable and shortlived changes in the
environment. {[}Review some lit{]} It is therefore likely that the
direction of selection on stomatal response rate varies depending on the
predictability and duration of environmental fluctuations, as well as
the energetic costs of stomatal movements.

What traits respond to selection on stomatal kinetics? We consider two
interconnected hypotheses that have thus far been treated in isolation.
The first hypothesis is that smaller guard cells open and close faster
because of their intrinsically greater surface area to volume ration (1,
2). For approximately cylindrical cell geometries, like that found in
guard cells (citation on guard cell shape?), surface area increases
linearly with radius, whereas volume increases in proportion to the
radius squared. Consequently, larger guard cells will require more time
to pump enough ions (wc?) to achieve a given change in turgor pressure
and, hence, stomatal pore aperture. A plant with large guard cells can
partially compensate for this geometric constraint by increasing the
density of transporters (wc?) in the guard cell membrane (3) (haworth et
al 2023 have some cites about this). However, this may be limited by the
available membrane area and/or energetic costs of maintaining high
transporter densities. Empirical support for the hypothesis that smaller
guard cells are faster is mixed. Some studies have found negative
correlations between guard cell size and stomatal response rate (1)
(cites), whereas others have found no relationship (cites). (apparently
hetherington and woodward 2003 mention size-speed relationship too -
check that. Lawson and Blatt 2014 throw shade on size being
deterministic)

One factor that might complicate the relationship between size and speed
is aperture. Stomatal aperture can vary from near 0 when guard cell
turgor pressure is low and asymptotically approach gmax as guard cell
turgor pressure approaches \(\infty\) (Figure 1X). The nonlinear
relationship between guard cell turgor and pore aperture implies that if
rate of change in turgor change is constant (\(\dv{P}{t} = C\)), the
rate of change in aperture, and hence stomatal conductance, will vary
depending on the initial aperture. When initial aperture is high,
stomatal conductance will respond more slowly than when initial aperture
is low. Since we generally do not observe individual stomatal aperture,
extending this idea to macroscale phenomena requires scaling by stomatal
density. We will use the term ``fraction of anatomical maximum stomatal
conductance'', symbolized as \fgmax. This value should be proportional
to the average aperture divided by its maximum aperture for any
arbitrary stomatal density {[}could a figure show how this interacts
with stomatal density?{]}. Leaves operating at fgmax closer to unity
will, all else being equal, will be slower to respond than leaves
operating with a \fgmax close to zero, independent of stomatal density.
(4) hypothesized that selection on stomatal density would maintain
typical operational stomatal conductance (gop) relative to gmax at a
sufficiently low value that stomatal conductance would be sensitive to
relatively small changes in guard cell turgor pressure. Consistent will
this hypothesis, gop:gmax is often near 0.25 (5, 6), well below
\(f_\text{gmax} = 1\), and in a range where \gsw would be responsive to
small changes in guard cell turgor pressure.

Putting these two hypotheses together, we predict that both guard cell
size and fgmax will influence stomatal kinetics, but possibly at
different scales of biological organization. Guard cell size tends to
vary less than stomatal density or aperture at many biological scales.
For example, on a mature leaf, guard cell size is essentially fixed
other than small changes in volume caused by turgor pressure (cite).
Compared to stomatal density, guard cell size is also less
developmentally plastic varies less genetically within species (cites).
If there is relatively little plastic or genetic variation among
individuals within a species, then most of the variance in stomatal
kinetics within species cannot be explained by variation in stomatal
size. In contrast, \fgmax varies immensely within and among individuals
because stomatal aperture responds dynamically over the day and in
response to environmental variation. Within a single leaf, \fgmax will
change over the course of a day in response to light signalling, change
in VPD, and starch accumulation (or sink strength? cite). During the
life a leaf, \fgmax will change in response to long-term stresses such
as drought. Because guard cell size and aperture typically vary at
different levels of biological organization, we predict that guard cell
size likely explains more variation in stomatal kinetics among species,
whereas \fgmax will explain more variation within species. Within
species variation can arise from either genetic or environmental
differences between individuals. Neither of these traits along will
determine all or even most of the variance in stomatal kinetics, which
is a complex response to many internal and external signals (cites).

Here we extend recent advances in phylogenetic comparative methods
(7--9) to test these predictions using data on stomatal closure kinetics
in response to an abrupt increase in VPD among 29 diverse wild tomato
populations in the genus \emph{Solanum}. We leveraged natural variation
in guard cell size among and within species to test whether leaves with
smaller guard cells close faster. We induced variation in fgmax through
a combination of growth and measurement light intensity. Growth light
intensity caused developmental plasticity in stomatal density that, when
crossed factorially with measurement light intensity, resulted in
variation in fgmax. By measuring multiple individuals from multiple
treatments across multiple species enabled us to estimate the effect of
guard cell size and \fgmax on stomatal kinetics and partition their
significance within and among species.

{[}not sure where this goes{]} many, large - slow because of large size,
but fast because of low gop few, small - slow because of high gop, but
fast because of small size some, moderate - best of both worlds? IDEA
FOR FIGURE - use empircal estimates to predict tau (z-axis) for a given
optimal gs as a function of

Is there a connection here to size-density scaling? It's an empirical
regularity, not a law of nature that higher gmax is associated with
smaller stomata. This is an observation that requires explanation, not a
presumption of truth.

What's never been done before is to put these two ideas together. This
means that weak relationships between size and speed might be because
people have not controlled for aperture (gop:gmax ratio).

Factors cannot also manifest at different scales of biological
organization. Here we focus on two levels: variation among species and
variation among individuals within species. To consider it abstractly,
if factors A and B influnce trait C according to some mechanistic model,
but factor A is typically fixed within species but varies among species,
then B will explain much of the variation in C within species. But if A
varies among species, then A will explain much of the variation in C
among species. We hypothesize that stomatal size is relatively constant
within species, but varies among species. In contrast, aperture
(gop:gmax) varies because of mismatches between leaf anatomy and gopt.

And that smaller stomata are faster.

These are related because the benefit of small stomata goes away if gop
is really high.

hooks:

Haworth et al (2018) paper arguing faster adaxial kinetics (I think).
Check other cites in Woning and Horak

Stomatal kinetics are important for optimizing response to rapid
fluctuations in light, leaf temperture, and VPD

curve shape does not match what you'd expect for constant alpha term

Anatomy possibly matters: shape and size

Adaxial stomata might close faster (why?)

What we did:\\
1. model stomatal close with hysteresis (can it explain overshoot and
wrong-way response?

\begin{enumerate}
\def\labelenumi{\arabic{enumi}.}
\setcounter{enumi}{1}
\tightlist
\item
  compare tau in tomatoes to that in rice
\item
  test for effect of size
\item
  test if adaxial+abaxial is faster than abaxial alone
\end{enumerate}

Phenomenologically, the following equation describes how stomatal
conductance responds to step changes in light intensity across dozens of
plant species (2):

\begin{equation}\phantomsection\label{eq-weibull}{
  g_\text{sw} = g_\text{f} + (g_\text{i} - g_\text{f}) e^{-\left(\frac{t}{\tau}\right)^\lambda}.
}\end{equation}

In this equation, \gsw is stomatal conductance at time \(t\),
\(g_\text{i}\) is the initial \gsw before the step change,
\(g_\text{f}\) is the final \gsw after the step change, \(\tau\) is a
time constant that describes how quickly stomata respond to the step
change, and \(\lambda\) is a shape parameter that describes how the
response rate changes over time. When \(\lambda = 1\), the response rate
is constant over time and \(\tau\) represents the time required for
\gsw to reach 36.8\% of the way between initial and final values. When
\(\lambda > 1\), the response rate increases over time, and when
\(\lambda < 1\), the response rate decreases over time. (2) refer to
\(\lambda\) as a lag-time parameter because empirically \(\lambda > 1\).
We find the same pattern for responses to a step change in VPD (see
results) and therefore adopt their terminology. We refer to stomatal
kinetic parameters \(\tau\) (time-constant) and \(\lambda\) (lag-time)
henceforth.

\subsection{Methods}\label{methods}

\subsubsection{Populations}\label{populations}

We analyzed stomatal responses to changing humidity among 29
ecologically diverse populations of wild tomato, including
representatives of all described species of \emph{Solanum} sect.
\emph{Lycopersicon} and sect. \emph{Lycopersicoides} (10) and the
cultivated tomato \emph{S. lycopersicum} var. \emph{lycopersicum}
(Table~\ref{tbl-populations}), as described in (11). These populations
were selected because they encompass the breadth of climatic variation
in the wild tomato clade (12). In one case, we substituted a congeneric
population of the species (\emph{S. galapagense}) because of difficulty
growing the focal population and we also added a population of \emph{S.
pennellii}. Due to constraints on growth space and time, we spread out
measurements over 61.1 weeks from August 29, 2022 to October 31, 2023.
Replicates within population were evenly spread out over this period to
prevent confounding of temporal variation in growth conditions with
variation among populations.

\begin{longtable}{>{\raggedright\arraybackslash}p{3.25cm}lrrr}

\caption{\label{tbl-populations}Accession information of \emph{Solanum}
populations used in this study. The species name, accession number,
collection latitude, longitude, and elevation. TGRC: Tomato Genetics
Resource Center.}

\tabularnewline

\toprule
Species & TGRC accession & Latitude & Longitude & Elevation (mas)\\
\midrule
\endfirsthead
\multicolumn{5}{@{}l}{\textit{(continued)}}\\
\toprule
Species & TGRC accession & Latitude & Longitude & Elevation (mas)\\
\midrule
\endhead

\endfoot
\bottomrule
\endlastfoot
\em{\cellcolor{gray!10}{S. arcanum}} & \cellcolor{gray!10}{LA2172} & \cellcolor{gray!10}{-6.008} & \cellcolor{gray!10}{-78.858} & \cellcolor{gray!10}{662}\\
\em{S. cheesmaniae} & LA0429 & -0.644 & -90.329 & 800\\
\em{\cellcolor{gray!10}{S. cheesmaniae}} & \cellcolor{gray!10}{LA3124} & \cellcolor{gray!10}{-0.804} & \cellcolor{gray!10}{-90.042} & \cellcolor{gray!10}{1}\\
\em{S. chilense} & LA1782 & -15.267 & -74.633 & 1000\\
\em{\cellcolor{gray!10}{S. chilense}} & \cellcolor{gray!10}{LA4117A} & \cellcolor{gray!10}{-22.907} & \cellcolor{gray!10}{-67.941} & \cellcolor{gray!10}{3540}\\
\addlinespace
\em{S. chmielewskii} & LA1028 & -13.883 & -73.017 & 3000\\
\em{\cellcolor{gray!10}{S. chmielewskii}} & \cellcolor{gray!10}{LA1316} & \cellcolor{gray!10}{-13.400} & \cellcolor{gray!10}{-73.906} & \cellcolor{gray!10}{2920}\\
\em{S. corneliomulleri} & LA0107 & -13.117 & -76.383 & 60\\
\em{\cellcolor{gray!10}{S. corneliomulleri}} & \cellcolor{gray!10}{LA0444} & \cellcolor{gray!10}{-13.433} & \cellcolor{gray!10}{-76.133} & \cellcolor{gray!10}{100}\\
\em{S. galapagense} & LA0436 & -0.953 & -90.978 & 40\\
\addlinespace
\em{\cellcolor{gray!10}{S. galapagense}} & \cellcolor{gray!10}{LA1044} & \cellcolor{gray!10}{-0.284} & \cellcolor{gray!10}{-90.548} & \cellcolor{gray!10}{0}\\
\em{S. habrochaites} & LA0407 & -2.181 & -79.884 & 70\\
\em{\cellcolor{gray!10}{S. habrochaites}} & \cellcolor{gray!10}{LA1777} & \cellcolor{gray!10}{-9.550} & \cellcolor{gray!10}{-77.700} & \cellcolor{gray!10}{3216}\\
\em{S. huaylasense} & LA1358 & -9.533 & -77.967 & 750\\
\em{\cellcolor{gray!10}{S. huaylasense}} & \cellcolor{gray!10}{LA1360} & \cellcolor{gray!10}{-9.546} & \cellcolor{gray!10}{-77.929} & \cellcolor{gray!10}{1490}\\
\addlinespace
\em{S. huaylasense} & LA1364 & -10.133 & -77.383 & 2920\\
\em{\cellcolor{gray!10}{S. lycopersicoides}} & \cellcolor{gray!10}{LA2951} & \cellcolor{gray!10}{-19.317} & \cellcolor{gray!10}{-69.450} & \cellcolor{gray!10}{2200}\\
\em{S. lycopersicoides} & LA4126 & -19.287 & -69.396 & 3120\\
\em{\cellcolor{gray!10}{S. neorickii}} & \cellcolor{gray!10}{LA1322} & \cellcolor{gray!10}{-13.483} & \cellcolor{gray!10}{-72.442} & \cellcolor{gray!10}{2380}\\
\em{S. neorickii} & LA2133 & -3.400 & -79.183 & 1980\\
\addlinespace
\em{\cellcolor{gray!10}{S. pennellii}} & \cellcolor{gray!10}{LA0716} & \cellcolor{gray!10}{-16.225} & \cellcolor{gray!10}{-73.617} & \cellcolor{gray!10}{50}\\
\em{S. pennellii} & LA0750 & -14.775 & -75.034 & 550\\
\em{\cellcolor{gray!10}{S. pennellii}} & \cellcolor{gray!10}{LA3778} & \cellcolor{gray!10}{-14.775} & \cellcolor{gray!10}{-75.034} & \cellcolor{gray!10}{616}\\
\em{S. peruvianum} & LA2744 & -18.550 & -70.150 & 400\\
\em{\cellcolor{gray!10}{S. peruvianum}} & \cellcolor{gray!10}{LA2964} & \cellcolor{gray!10}{-18.028} & \cellcolor{gray!10}{-70.835} & \cellcolor{gray!10}{75}\\
\addlinespace
\em{S. pimpinellifolium} & LA1269 & -11.483 & -77.075 & 400\\
\em{\cellcolor{gray!10}{S. pimpinellifolium}} & \cellcolor{gray!10}{LA1589} & \cellcolor{gray!10}{-8.433} & \cellcolor{gray!10}{-78.817} & \cellcolor{gray!10}{30}\\
\em{S. pimpinellifolium} & LA2933 & -1.442 & -80.562 & 375\\
\em{\cellcolor{gray!10}{S. sitiens}} & \cellcolor{gray!10}{LA4116} & \cellcolor{gray!10}{-22.159} & \cellcolor{gray!10}{-68.782} & \cellcolor{gray!10}{2960}\\*

\end{longtable}

\subsubsection{Plant growth conditions and light
treatments}\label{plant-growth-conditions-and-light-treatments}

A thorough description of plant growth conditions can be found (11),
therefore we summarize the most important information here. Seeds
provided by the Tomato Genetics Resource Center germinated on moist
paper in plastic boxes after soaking for 30-60 minutes in a 50\% (volume
per volume) solution of household bleach and water, followed by a
thorough rinse. We transferred seedlings to cell-pack flats containing
Pro-Mix BX potting mix (Premier Tech, Rivière-du-Loup, Quebec, Canada)
once cotyledons fully emerged, typically within 1-2 weeks of sowing. We
grew seeds and seedlings for both sun and shade treatments under the
same environmental conditions (12:12 h,
24.3:\(\SI{21.7}{\degreeCelsius}\), 49.6:58.4 RH day:night cycle). LED
light provided
\(\mathrm{PPFD} = \SI{267}{\micro\mole\per\meter\squared\per\second}\)
(Fluence RAZRx, Austin, Texas, USA) during the germination and seedling
stages.

Immediately prior to starting growth light intensity treatments (sun and
shade), we transplanted seedlings to 3.78 L plastic pots containing 60\%
Pro-Mix BX potting mix, 20\% coral sand (Pro-Pak, Honolulu, Hawaiʻi,
USA), and 20\% cinders (Niu Nursery, Honolulu, Hawaiʻi, USA) withslow
release NPK fertilizer following manufacturer instructions (Osmocote
Smart-Release Plant Food Flower \& Vegetable, The Scotts Company,
Marysville, Ohio, USA). Percentage composition is on a volume basis. We
watered to field capacity three times per week to prevent drought
stress. Seedlings were randomly assigned in alternating order within
population to the sun or shade treatment during transplanting. The
average daytime \ppfd{} was \ppfdqty{761} and \ppfdqty{115} for sun and
shade treatments, respectively.

\subsubsection{Stomatal responses to changing
humidity}\label{stomatal-responses-to-changing-humidity}

We selected a fully expanded, unshaded leaf at least six leaves above
the cotyledons during early vegetative growth. This typically meant that
plants had grown in light treatments for \(\approx\) 4 weeks, ensuring
they had time to sense and respond developmentally to the light
intensity of the treatment rather than the seedling conditions (13). See
(11) for further detail.

To measure stomatal responses to a step change in humidity, we first
acclimated the focal leaf to high humidity, rapidly decreased the
humidity, and logged \gsw through time. We refer to this as a
humidity-response curve. We measured humidity-response curves using a
portable infrared gas analyzer (LI-6800PF, LI-COR Biosciences, Lincoln,
Nebraska, USA). Light-acclimated plants were placed under LEDs dimmed to
match their light treatment during gas exchange measurements. We
estimated \gsw at ambient CO\(_2\) (\caequals{415}) and
\tleafequals{25.0}.

We collected four humidity-response curves per leaf, an amphi
(untreated) curve and a pseudohypo (treated) curve at high
light-intensity (\ppfdequals{2000}; 97.8:2.24 red:blue) and low
light-intensity (\ppfdequals{150}; 87.0:13.0 red:blue). We always
measured high light-intensity curves first because photosynthetic
downregulation is faster than upregulation in these species.
Pseudohypostomatous leaves are the same as the amphistomatous leaves
except gas exchange through the upper (adaxial) surface blocked by a
neutral density plastic (propafilm). Comparing amphi and pseudohypo
curves on the same leaf enables us to test whether stomatal kinetics are
different for ab- and adaxial stomata on the same leaf. To compensate
for reduced transmission, we increased incident \ppfd{} for pseudohypo
leaves by a factor 1/0.91, the inverse of the measured transmissivity of
the propafilm. We also set the stomatal conductance ratio, for purposes
of calculating boundary layer conductance, to 0 for pseudohypo leaves
following manufacturer directions.

To control for order effects, we alternated between starting with amphi
or pseudohypo leaf measurements. We made measurements over two days. On
the first day, we measured high and low light-intensity curves for
either amphi or pseudohypo leaves; on the second day, we measured high
and low light-intensity curves on the other leaf type. The irradiance of
the light source in the pseudohypo leaf was higher because the propafilm
reduces transmission.

In all cases, we acclimated the focal leaf to high light
(\ppfdequals{2000}) and high relative humidity (\rhequals{70}) until
\gsw{} reachws its maximum. After that, we decreased \rh{} to
\(\approx 10\%\) to induce rapid stomatal closure. All other
environmental conditions in the leaf chamber remained the same. We
started logging data after the transient ``wrong-way response''
(citation) and continued logging data until \gsw{} reached its nadir. We
then acclimated the leaf to low light (\ppfdequals{150}) and
\rhequals{70} before inducing stomatal closure with low \rh{} and
logging data as described above.

\subsubsection{Hypothesis generation and
testing}\label{hypothesis-generation-and-testing}

The primary aims and \emph{a priori} hypotheses of this study are
described in (11). The idea to use \gsw response curves to test
hypotheses about which traits influence stomatal kinetics came after the
data had been collected, but not yet analyzed. We started with two
\emph{a priori} hypotheses based on a recent metaanalysis of stomatal
responses to light (2). We hypothesized that 1) guard cell size would
influence \(\tau\) and 2) that \(\tau\) would be greater on the
abaxial-only leaves. We planned to compare relationships within and
among populations using phylogenetic comparative methods. We made the
decision to test for an association between \fgmax and \(\tau\) after
exploratory analyses revealed a potential association. Adding hypotheses
after some results have been analyzed, also known as ``Hypothesizing
After Results Known'' (HARKing), can lead to reports of spurious
associations that are not repeatable in later studies (14, 15).
Therefore, our conclusions about the effect of \fgmax on stomatal
kinetics should be interpreted with caution until they are replicated.

We estimated \(\tau\) and \(\lambda\) for each response curve by fitting
Equation~\ref{eq-weibull} to a time-series of \gsw values using Bayesian
nonlinear regression. WORKING HERE

We fit multiresponse Bayesian mixed effects models with phylogenetically
structured random effects to answer the following core questions:

\begin{enumerate}
\def\labelenumi{\arabic{enumi}.}
\tightlist
\item
  Are \gcl and \fgmax positively associated with \(\tau\)?
\item
  At what biological levels of organization do \gcl and \fgmax influence
  \(\tau\)?
\item
  Are abaxial-only stomatal responses slower than whole leaf stomatal
  responses?
\end{enumerate}

SCRAPS THAT I NEED TO WEAVE INTO MS

We used a combination of model selection and parameter estimation to
evaluate these hypotheses. All statistical models included fixed effects
of manipulative treatments: growth light intensity (sun and shade);
measurement light intensity (high and low); and curve type
(amphistomatous and pseudohypostomatous). All models also included
population as a phylogenetically structured random effect.

We accounted for uncertainty in \(\tau\) and \(\lambda\) using the
standard deviation of the posterior distribution of each estimate (see
above).

We used the RAxML whole-transcriptome concatenated phylogeny based on
2,745 100-kb genomic windows from (12). Two of our populations were not
in this tree. We used accession LA1044 in place of LA3909, two
populations of \emph{S. galapagense}; LA0750 was added as sister to
LA0716, two closely related populations of \emph{S. pennellii}. The node
separating LA0750 from LA0716 was placed half-way between the next
deepest node.

\subsubsection{inferring different levels of
effects??}\label{inferring-different-levels-of-effects}

We analyzed three nonmutually exclusive paths through which \fgmax and
\gcl could influence stomatal kinetic parameters \(\lambda\) and
\(\tau\). First, we inferred phylogenetic effects by estimating the
phylogenetic correlation among populations between anatomical
(\fgmax and \gcl) and kinetic (\(\lambda\) and \(\tau\)) variables. To
test for causation, we used the phylogenetic covariance matrix to
estimate partial correlations between anatomical and kinetic variables
(9). We interpret a significant partial correlation between variables as
a plausible causal hypothesis. Second, we inferred individual-level
effects of \fgmax and \gcl on kinetic parameters using linear multiple
regression that simultaneously accounts for phylogenetic and treatment
effects. We used the leave-one-out cross-validation information
criterion (LOOIC) to compare the fit of models using the \emph{R}
package \textbf{loo} version 2.9.0 (16) to calculate LOOIC values. We
compared models with all permutations of \fgmax and \gcl influencing
\(\lambda\) and \(\tau\). We selected the model with the lowest LOOIC
value (\textbf{?@tbl-comparison}) to generate posterior predictions for
hypothesis testing. If an effect was included in the best model (lowest
LOOIC), then we considered whether the 95\% confidence interval
overlapped zero to ascertain a significant effect.

THIRD: mediating treatment effects (plasticity)

\subsubsection{Variance decomposition and heritability
(somewhere)}\label{variance-decomposition-and-heritability-somewhere}

We decomposed the variance in log-transformed response variables
(\(\lambda\), \(tau\), \fgmax, \gcl) into phylogenetic, population,
between-individual, and within-individual levels. The phylogenetic
variance component quantifies trait evolution between populations
congruent with the phylogenetic relationships, whereas the population
component quantifies variance among populations independent of
phylogenetic relationships (17, 18). The between- and within-individual
variance components quantify variance among individuals within the same
population and treatment. The within-individual component was estimated
from repeated measures of the same leaf at different light intensities.
It was not possible to estimate within-individual variance in guard cell
length with our study design because we only measured average guard cell
length once per leaf. We estimated the between-individual variance from
the residual variance, which also includes measurement error. The
phylogenetic heritability is equivalent to the phylogenetic variance
component and is estimated following (17); (18) as:

\[h^2_{\mathrm{phy}} = \frac{\sigma^2_\mathrm{phy}}{\sigma^2_\mathrm{phy} + \sigma^2_\mathrm{pop} + \sigma^2_\mathrm{between} + \sigma^2_\mathrm{within}},\]

where \(\sigma^2_\mathrm{phy}\), \(\sigma^2_\mathrm{pop}\),
\(\sigma^2_\mathrm{between}\), and \(\sigma^2_\mathrm{within}\) are the
phylogenetic, population, between-individual, and within-individual
variance components, respectively.

\subsubsection{Model comparison
(somewhere)}\label{model-comparison-somewhere}

add info

\subsection{Results}\label{results}

Stomatal conductance (\gsw) decrease rapidly in response to a step
increase in VPD. Consider a typical leaf in the experiment possessing
the median time constant (\(\tau\)) and lag time (\(\lambda\)) among
wild tomato accessions in all treatments. After VPD increased the
transient wrong way response elapsed, it took 121 s for \gsw to decrease
half-way (\(t_{50}\)) from its initial to final steady state value. In
terms of the kinetic model parameters, most variation among leaves was
due to difference in the time constant rather than the lag time. In the
previous example, increasing \(\lambda\) from its median to its maximum
estimated value among accessions increased the \(t_{50}\) by only 8.2 s.
In contrast, increasing \(\tau\) from its median to maximum value
increased \(t_{50}\) by 129 s

Growth light intensity did not significantly affect \(\tau\)
Figure~\ref{fig-accession-kinetics}. In sun plants, \(\tau\) was on
average 6.46\% {[}95\% CI: 1.64 to 11.6\%{]} lower than in shade plants
after accounting for other explanatory variables. Within the same leaf,
increasing the measurement light intensity prior to the VPD step change
from \ppfdqty{150} to \ppfdqty{2000} significantly increased \(\tau\) by
14.9\% {[}95\% CI: 8.91 to 21.8\%{]}
Figure~\ref{fig-accession-kinetics}. The time lag \(\lambda\) responded
to growth, but not measurement light intensity after accounting for
other explanatory variables Figure~\ref{fig-accession-kinetics}. In sun
plants, \(\tau\) was on average 11\% {[}95\% CI: 6.31 to 15.6\%{]}
greater than in shade plants. Higher measurement light decreased
increased \(\lambda\) slightly by 0.255\% {[}95\% CI: -2.18 to
2.89\%{]}.

\begin{itemize}
\tightlist
\item
  table of symbols?
\item
  table of model estimates for tau and lambda
\end{itemize}

\begin{figure}

\centering{

\includegraphics{../figures/accession-kinetics.pdf}

}

\caption{\label{fig-accession-kinetics}\textbf{Stomatal kinetics parameters $\lambda$ (lag time; top facets) and $\tau$ (time constant; lower facets) vary among wild tomato populations.}
Stomatal conductance decreases faster (lower \(\tau\)) in response to a
step change in vapor pressure deficit (VPD) in leaves when measured
under low light intensity. The pattern was consistent for both shade
(left facets) and sun (right facets) grown plants. The pattern for
\(\lambda\)\$ was qualitatively similar to that for \(\tau\). Each point
is the average parameter value for one accession in that treatment
combination. The growth and measurement light intensity treatments are
described in the Materials and Methods section.}

\end{figure}%

\begin{longtable*}[t]{lrr}
\toprule
Species & mean & sd\\
\midrule
setosa & 5.006 & 0.3524897\\
versicolor & 5.936 & 0.5161711\\
virginica & 6.588 & 0.6358796\\
\bottomrule
\end{longtable*}

Among the four traits in our model (\fgmax, \gcl, \(\lambda\), and
\(\tau\)), the preponderance of the variance was typically within or
between individuals after accounting for treatment effects
Figure~\ref{fig-variance}, Table~\ref{tbl-variance}. The exception to
this pattern was \gcl{}, for which the phylogenetic component (aka
phylogenetic heritability) was larger Figure~\ref{fig-variance}. The
population component independent of phylogeny explained little of the
trait variation. The between-individual variance was usually high, but
this component confounds true between-individual variance with
measurement error, so should be interpreted with caution.

\begin{figure}

\centering{

\includegraphics{../figures/variance.pdf}

}

\caption{\label{fig-variance}\textbf{foo} bar.}

\end{figure}%

\begin{longtable}{lll}

\caption{\label{tbl-variance}Variance components of stomatal kinetic
traits. The phylogenetic component is equivalent to the phylogenetic
heritability. The population component is the nonphylogenetic variance
among populations. Between- and within-individual variance components
are among individuals within a population after accounting for treatment
effect. Estimates and 95\% confidence intervals (CI) are estimated as
the median and quantile intervals of the posterior distribution.
\(f_\text{gmax}\): stomatal conductance as a fraction of anatomical
maximum stomatal conductance; \(l_\text{gc}\): guard cell length;
\(\lambda\): lag time; \(\tau\): time constant.}

\tabularnewline

\toprule
component & \% variance & 95\% CI\\
\midrule
\endfirsthead
\multicolumn{3}{@{}l}{\textit{(continued)}}\\
\toprule
component & \% variance & 95\% CI\\
\midrule
\endhead

\endfoot
\bottomrule
\endlastfoot
\addlinespace[0.3em]
\multicolumn{3}{l}{\textbf{$\log(f_\mathrm{gmax})$}}\\
\hspace{1em}phylogenetic & 13.9 & {}[3.2, 30.2]\\
\hspace{1em}population & 5.4 & {}[1.3, 13.0]\\
\hspace{1em}between-individual & 40.2 & {}[33.5, 45.5]\\
\hspace{1em}within-individual & 39.4 & {}[31.6, 45.2]\\
\addlinespace[0.3em]
\multicolumn{3}{l}{\textbf{$\log(l_\mathrm{gc})$}}\\
\hspace{1em}\hspace{1em}phylogenetic & 57.0 & {}[26.9, 76.0]\\
\hspace{1em}\hspace{1em}population & 5.3 & {}[0.4, 22.9]\\
\hspace{1em}\hspace{1em}between-individual & 37.3 & {}[22.6, 54.1]\\
\addlinespace[0.3em]
\multicolumn{3}{l}{\textbf{$\log(\lambda)$}}\\
\hspace{1em}phylogenetic & 15.5 & {}[3.9, 30.3]\\
\hspace{1em}population & 1.6 & {}[0.0, 7.7]\\
\hspace{1em}between-individual & 61.3 & {}[50.9, 68.7]\\
\hspace{1em}within-individual & 20.6 & {}[16.1, 25.1]\\
\addlinespace[0.3em]
\multicolumn{3}{l}{\textbf{$\log(\tau)$}}\\
\hspace{1em}phylogenetic & 34.6 & {}[17.2, 51.3]\\
\hspace{1em}population & 1.0 & {}[0.0, 7.4]\\
\hspace{1em}between-individual & 40.1 & {}[30.3, 49.4]\\
\hspace{1em}within-individual & 23.5 & {}[17.2, 29.8]\\*

\end{longtable}

Guard cell size and \fgmax affected stomatal kinetics, but at different
levels of biological organization. In the model with the lowest LOOIC
(see \textbf{?@tbl-comparison} for model comparison results), the only
significant partial correlation was between \gcl and \(\tau\)
(\(\rho = 0.77\) {[}95\% CI: -0.1 to 0.98\$) was a significant
phylogenetic correlation between \gcl and \(\tau\) (\textbf{fig?}-?).
PARTIAL CORRELATION RESULT. Variation in \gcl within populations did not
explain within-population variation in \(\tau\) Table~\ref{tbl-fit}.
NOTE THAT BEST MODEL RETAINED INDIVIDUAL LEVEL EFFECT OF GCL, BUT
COEFFICIENT WAS NOT SIGNIFICANT. OTHER BEST MODELS DID NOT RETAIN THIS.
DID GCL MEDIATE AFFECT OF TREATMENTS?

Need to show (this is a growing, incomplete list): - model comparison
results table - figure of tau versus gcl per accession - phylogenetic
partial correlation - partial correlation showing that gcl and tau are
(only?) significant phylogenetic covariates

\begin{longtable}[t]{>{\raggedright\arraybackslash}p{4in}>{\raggedright\arraybackslash}p{2in}}

\caption{\label{tbl-fit}Fixed effects (posterior mean, SE, and 95\%
confidence intervals).}

\tabularnewline

\\
\toprule
\textbf{Parameter} & \textbf{Estimate [95\% CI]}\\
\midrule
\endfirsthead
\caption[]{Fixed effects (posterior mean, SE, and 95\% confidence intervals). \textit{(continued)}}\\
\toprule
\textbf{Parameter} & \textbf{Estimate [95\% CI]}\\
\midrule
\endhead

\endfoot
\bottomrule
\endlastfoot
\addlinespace[0.3em]
\multicolumn{2}{l}{\textbf{$\log(\lambda)$}}\\
\hspace{1em}effect of $\log(f_g)$ on $\log(\lambda)$ & 0.12 [0.09, 0.14]\\
\hspace{1em}effect of $\log(l_\mathrm{gc})$ on $\log(\lambda)$ & 0.13 [-0.20, 0.47]\\
\hspace{1em}effect of curvetypepseudohypo on $\log(\lambda)$ & 0.05 [0.02, 0.07]\\
\hspace{1em}effect of high measurement light intensity on $\log(\lambda)$ & 0.00 [-0.02, 0.03]\\
\hspace{1em}intercept (shade, low light) & 0.06 [-0.96, 1.00]\\
\hspace{1em}effect of sun growth treatment on $\log(\lambda)$ & 0.10 [0.06, 0.14]\\
\addlinespace[0.3em]
\multicolumn{2}{l}{\textbf{$\log(\tau)$}}\\
\hspace{1em}effect of $\log(f_g)$ on $\log(\tau)$ & 0.22 [0.15, 0.28]\\
\hspace{1em}effect of curvetypepseudohypo on $\log(\tau)$ & 0.16 [0.11, 0.21]\\
\hspace{1em}effect of high measurement light intensity on $\log(\tau)$ & 0.14 [0.09, 0.20]\\
\hspace{1em}intercept (shade, low light) & 5.60 [5.22, 5.92]\\
\hspace{1em}effect of sun growth treatment on $\log(\tau)$ & 0.06 [0.02, 0.11]\\
\addlinespace[0.3em]
\multicolumn{2}{l}{\textbf{$\log(f_g)$}}\\
\hspace{1em}effect of curvetypepseudohypo on $\log(f_g)$ & -0.62 [-0.65, -0.58]\\
\hspace{1em}effect of high measurement light intensity on $\log(f_g)$ & 0.78 [0.75, 0.81]\\
\hspace{1em}intercept (shade, low light) & -2.58 [-2.83, -2.29]\\
\hspace{1em}effect of sun growth treatment on $\log(f_g)$ & -0.26 [-0.33, -0.20]\\
\addlinespace[0.3em]
\multicolumn{2}{l}{\textbf{$\log(l_\mathrm{gc})$}}\\
\hspace{1em}effect of curvetypepseudohypo on $\log(l_\mathrm{gc})$ & 0.06 [0.05, 0.07]\\
\hspace{1em}intercept (shade, low light) & 2.90 [2.74, 3.07]\\
\hspace{1em}effect of sun growth treatment on $\log(l_\mathrm{gc})$ & 0.12 [0.11, 0.13]\\*

\end{longtable}

\begin{table}

\caption{\label{tbl-foo}}

\centering{

\captionsetup{labelsep=none}

[!h]
\centering
\begin{tabular}[t]{lcccc}
\toprule
\multicolumn{1}{c}{ } & \multicolumn{2}{c}{$\lambda$} & \multicolumn{2}{c}{$\tau$} \\
\cmidrule(l{3pt}r{3pt}){2-3} \cmidrule(l{3pt}r{3pt}){4-5}
$\Delta \text{LOOIC}$ & \fgmax & \gcl & \fgmax & \gcl\\
\midrule
\cellcolor{gray!10}{0.00} & \cellcolor{gray!10}{\cmark} & \cellcolor{gray!10}{\cmark} & \cellcolor{gray!10}{\cmark} & \cellcolor{gray!10}{}\\
4.34 & \cmark &  & \cmark & \cmark\\
\cellcolor{gray!10}{5.63} & \cellcolor{gray!10}{\cmark} & \cellcolor{gray!10}{} & \cellcolor{gray!10}{\cmark} & \cellcolor{gray!10}{}\\
6.69 & \cmark & \cmark & \cmark & \cmark\\
\cellcolor{gray!10}{20.89} & \cellcolor{gray!10}{\cmark} & \cellcolor{gray!10}{} & \cellcolor{gray!10}{} & \cellcolor{gray!10}{\cmark}\\
\addlinespace
21.65 & \cmark & \cmark &  & \cmark\\
\cellcolor{gray!10}{26.14} & \cellcolor{gray!10}{\cmark} & \cellcolor{gray!10}{\cmark} & \cellcolor{gray!10}{} & \cellcolor{gray!10}{}\\
30.03 & \cmark &  &  & \\
\cellcolor{gray!10}{44.71} & \cellcolor{gray!10}{} & \cellcolor{gray!10}{\cmark} & \cellcolor{gray!10}{\cmark} & \cellcolor{gray!10}{}\\
48.42 &  &  & \cmark & \\
\addlinespace
\cellcolor{gray!10}{49.58} & \cellcolor{gray!10}{} & \cellcolor{gray!10}{} & \cellcolor{gray!10}{\cmark} & \cellcolor{gray!10}{\cmark}\\
54.02 &  & \cmark & \cmark & \cmark\\
\cellcolor{gray!10}{69.10} & \cellcolor{gray!10}{} & \cellcolor{gray!10}{\cmark} & \cellcolor{gray!10}{} & \cellcolor{gray!10}{\cmark}\\
77.31 &  &  &  & \\
\cellcolor{gray!10}{78.42} & \cellcolor{gray!10}{} & \cellcolor{gray!10}{} & \cellcolor{gray!10}{} & \cellcolor{gray!10}{\cmark}\\
\addlinespace
88.14 &  & \cmark &  & \\
\bottomrule
\end{tabular}

}

\end{table}%

\subsection{References}\label{references}

\phantomsection\label{refs}
\begin{CSLReferences}{0}{1}
\bibitem[\citeproctext]{ref-drake_smaller_2013}
\CSLLeftMargin{1. }%
\CSLRightInline{P. L. Drake, R. H. Froend, P. J. Franks,
\href{https://doi.org/10.1093/jxb/ers347}{Smaller, faster stomata:
Scaling of stomatal size, rate of response, and stomatal conductance}.
\emph{Journal of Experimental Botany} \textbf{64}, 495--505 (2013).}

\bibitem[\citeproctext]{ref-woning_revisiting_2026}
\CSLLeftMargin{2. }%
\CSLRightInline{N. Woning, \emph{et al.}, Revisiting the relationship
between stomatal size and speed across species -- a meta-analysis.
\emph{New Phytologist} (2026). \url{https://doi.org/10.1111/nph.70842}.}

\bibitem[\citeproctext]{ref-raven_speedy_2014}
\CSLLeftMargin{3. }%
\CSLRightInline{J. A. Raven,
\href{https://doi.org/10.1093/jxb/eru032}{Speedy small stomata?}
\emph{Journal of Experimental Botany} \textbf{65}, 1415--1424 (2014).}

\bibitem[\citeproctext]{ref-franks_physiological_2012}
\CSLLeftMargin{4. }%
\CSLRightInline{P. J. Franks, I. J. Leitch, E. M. Ruszala, A. M.
Hetherington, D. J. Beerling,
\href{https://doi.org/10.1098/rstb.2011.0270}{Physiological framework
for adaptation of stomata to {CO} \(_{\textrm{2}}\) from glacial to
future concentrations}. \emph{Philosophical Transactions of the Royal
Society B: Biological Sciences} \textbf{367}, 537--546 (2012).}

\bibitem[\citeproctext]{ref-mcelwain_using_2016}
\CSLLeftMargin{5. }%
\CSLRightInline{J. C. McElwain, C. Yiotis, T. Lawson,
\href{https://doi.org/10.1111/nph.13579}{Using modern plant trait
relationships between observed and theoretical maximum stomatal
conductance and vein density to examine patterns of plant
macroevolution}. \emph{New Phytologist} \textbf{209}, 94--103 (2016).}

\bibitem[\citeproctext]{ref-murray_consistent_2020}
\CSLLeftMargin{6. }%
\CSLRightInline{M. Murray, \emph{et al.},
\href{https://doi.org/10.1086/706260}{Consistent relationship between
field-measured stomatal conductance and theoretical maximum stomatal
conductance in {C}\(_{\textrm{3}}\) woody angiosperms in four major
biomes}. \emph{International Journal of Plant Sciences} \textbf{181},
142--154 (2020).}

\bibitem[\citeproctext]{ref-westoby_phylogenetically_2023}
\CSLLeftMargin{7. }%
\CSLRightInline{M. Westoby, L. Yates, B. Holland, B. Halliwell,
\href{https://doi.org/10.1111/1365-2745.14150}{Phylogenetically
conservative trait correlation: {Quantification} and interpretation}.
\emph{Journal of Ecology} \textbf{111}, 2105--2117 (2023).}

\bibitem[\citeproctext]{ref-muir_how_2023}
\CSLLeftMargin{8. }%
\CSLRightInline{C. D. Muir, \emph{et al.},
\href{https://doi.org/10.1086/723780}{How important are functional and
developmental constraints on phenotypic evolution? {An} empirical test
with the stomatal anatomy of flowering plants}. \emph{The American
Naturalist} \textbf{201}, 794--812 (2023).}

\bibitem[\citeproctext]{ref-halliwell_multiresponse_2025}
\CSLLeftMargin{9. }%
\CSLRightInline{B. Halliwell, B. R. Holland, L. A. Yates,
\href{https://doi.org/10.1111/brv.70001}{Multi‐response phylogenetic
mixed models: Concepts and application}. \emph{Biological Reviews}
\textbf{100}, 1294--1316 (2025).}

\bibitem[\citeproctext]{ref-peralta_taxonomy_2008}
\CSLLeftMargin{10. }%
\CSLRightInline{I. E. Peralta, D. M. Spooner, S. Knapp, Taxonomy of wild
tomatoes and their relatives (\emph{solanum} sect.
\emph{Lycopersicoides}, sect. \emph{Juglandifolia}, sect.
\emph{Lycopersicon}; {Solanaceae}). \textbf{84} (2008).}

\bibitem[\citeproctext]{ref-muir_plasticity_2025}
\CSLLeftMargin{11. }%
\CSLRightInline{C. D. Muir, W. S. Lim, D. Wang,
\href{https://doi.org/10.1101/2025.08.19.671015}{Plasticity and
adaptation to high light intensity amplify the advantage of
amphistomatous leaves}. (2025).}

\bibitem[\citeproctext]{ref-pease_phylogenomics_2016}
\CSLLeftMargin{12. }%
\CSLRightInline{J. B. Pease, D. C. Haak, M. W. Hahn, L. C. Moyle,
\href{https://doi.org/10.1371/journal.pbio.1002379}{Phylogenomics
reveals three sources of adaptive variation during a rapid radiation}.
\emph{PLOS Biology} \textbf{14}, e1002379 (2016).}

\bibitem[\citeproctext]{ref-schoch_dependence_1980}
\CSLLeftMargin{13. }%
\CSLRightInline{P.-G. Schoch, C. Zinsou, M. Sibi,
\href{https://doi.org/10.1093/jxb/31.5.1211}{Dependence of the stomatal
index on environmental factors during stomatal differentiation in leaves
of \emph{{Vigna} sinensis} {L}.: 1. {Effect} of light intensity}.
\emph{Journal of Experimental Botany} \textbf{31}, 1211--1216 (1980).}

\bibitem[\citeproctext]{ref-kerr_harking_1998}
\CSLLeftMargin{14. }%
\CSLRightInline{N. L. Kerr,
\href{https://doi.org/10.1207/s15327957pspr0203_4}{{HARKing}:
{Hypothesizing} {After} the {Results} are {Known}}. \emph{Personality
and Social Psychology Review} \textbf{2}, 196--217 (1998).}

\bibitem[\citeproctext]{ref-smaldino_natural_2016}
\CSLLeftMargin{15. }%
\CSLRightInline{P. E. Smaldino, R. McElreath,
\href{https://doi.org/10.1098/rsos.160384}{The natural selection of bad
science}. \emph{Royal Society Open Science} \textbf{3}, 160384 (2016).}

\bibitem[\citeproctext]{ref-vehtari_practical_2017}
\CSLLeftMargin{16. }%
\CSLRightInline{A. Vehtari, A. Gelman, J. Gabry,
\href{https://doi.org/10.1007/s11222-016-9696-4}{Practical {Bayesian}
model evaluation using leave-one-out cross-validation and {WAIC}}.
\emph{Statistics and Computing} \textbf{27}, 1413--1432 (2017).}

\bibitem[\citeproctext]{ref-lynch_methods_1991}
\CSLLeftMargin{17. }%
\CSLRightInline{M. Lynch,
\href{https://doi.org/10.1111/j.1558-5646.1991.tb04375.x}{Methods for
the analysis of comparative data in evolutionary biology}.
\emph{Evolution} \textbf{45}, 1065--1080 (1991).}

\bibitem[\citeproctext]{ref-garamszegi_general_2014}
\CSLLeftMargin{18. }%
\CSLRightInline{P. De Villemereuil, S. Nakagawa,
{``\href{https://doi.org/10.1007/978-3-662-43550-2_11}{General
{Quantitative} {Genetic} {Methods} for {Comparative} {Biology}}''} in
\emph{Modern {Phylogenetic} {Comparative} {Methods} and {Their}
{Application} in {Evolutionary} {Biology}}, L. Z. Garamszegi, Ed.
(Springer Berlin Heidelberg, 2014), pp. 287--303.}

\end{CSLReferences}

\newpage{}

\section{Supplementary Materials}\label{supplementary-materials}

\renewcommand{\thefigure}{S\arabic{figure}}
\renewcommand{\thetable}{S\arabic{table}}
\renewcommand{\theequation}{S\arabic{equation}}
\setcounter{figure}{0}
\setcounter{table}{0}
\setcounter{equation}{0}
\setcounter{page}{1}

\subsection{MANUSCRAPS}\label{manuscraps}

Note: this theory did not really pan out. It did not come up with models
that really fit better than the CD Weibull model and they included an
extra parameter. Therefore, I am tabling this for now and trying
OnGuard3 instead.

\subsubsection{Theory}\label{theory}

Our goal here is derive equations that integrate the effects of guard
cell size and aperture on stomatal kinetics. A fully mechanistic model
is beyond the scope of our goals here. Instead, we use phenomenological
equations that adequately capture empirical patterns. There are three
main assumptions of our model. First, we assume linear relaxation
kinetics for guard cell turgor pressure (\(P\)):

\[ \frac{d P}{d t} = \frac{P^* - P}{\tau}.\] In this differential
equation, \(P^*\) is the equilibrium turgor pressure and \(\tau\) is a
time constant that determines the speed of relaxation. Solving this
equation gives:

\[ P(t) = P^* + [P(0) - P^*] e^{-t/\tau}, \] where \(P(0)\) is the
initial turgor pressure at time \(t=0\).

The rate of relaxation is determined by the surface area to volume
ratio. Assume constant transport per surface area. Surface area
determined by geometry.

What is eqn for this?



\end{document}
